\section{Tool utilizzati}

\subsection{Maven}

Maven è uno strumento di gestione e automazione dei progetti software, ampiamente utilizzato nell’ecosistema Java. Permette di gestire le dipendenze, automatizzare la compilazione, l’esecuzione dei test e la creazione dei pacchetti eseguibili.

Nel progetto UninaFoodLab, Maven è stato utilizzato per semplificare la configurazione e la gestione delle librerie necessarie, come JavaFX per la realizzazione dell’interfaccia grafica e il driver JDBC per la connessione al database PostgreSQL. Tutte le dipendenze sono dichiarate nel file `pom.xml`, che consente di mantenere il progetto facilmente aggiornabile e portabile.

L’integrazione con JavaFX è stata gestita tramite le apposite dipendenze e plugin, permettendo di compilare ed eseguire l’applicazione con semplici comandi Maven. Questo approccio ha garantito una maggiore efficienza nello sviluppo e una migliore organizzazione del codice.


\subsection{JavaFX}

JavaFX è una libreria per la creazione di interfacce utente grafiche (GUI) in Java. È stata progettata per fornire un ambiente di sviluppo moderno e ricco di funzionalità, consentendo la creazione di applicazioni desktop e web con un aspetto e un comportamento coerenti.

Nel progetto UninaFoodLab, JavaFX è stato utilizzato per realizzare l'interfaccia grafica dell'applicazione. Grazie a JavaFX, è stato possibile implementare facilmente elementi UI complessi, come tabelle, grafici e controlli personalizzati, migliorando l'usabilità e l'estetica dell'applicazione.

L'integrazione di JavaFX con Maven ha semplificato ulteriormente il processo di sviluppo, consentendo di gestire le dipendenze e le configurazioni necessarie per l'utilizzo della libreria in modo efficiente e organizzato.

\subsection{PostgreSQL}

PostgreSQL è un sistema di gestione di database relazionali.

Nel progetto UninaFoodLab, PostgreSQL è stato scelto come database principale per la sua capacità di gestire grandi volumi di dati. L'integrazione con Java è stata realizzata tramite JDBC (Java Database Connectivity), che ha permesso di stabilire una connessione tra l'applicazione Java e il database PostgreSQL in modo semplice e diretto.

