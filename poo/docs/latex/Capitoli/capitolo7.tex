\subsection{JDBC}
La piattaforma UninaFoodLab utilizza JDBC (Java Database Connectivity) per gestire la connessione e le operazioni con il database PostgreSQL. JDBC è uno standard Java che permette di interagire con database relazionali tramite driver specifici, garantendo portabilità e flessibilità.

\subsubsection{Connessione al database}
La connessione al database è gestita tramite la classe \texttt{ConnectionJavaDb}, che centralizza la logica di accesso e configurazione. I parametri di connessione (URL, utente, password) sono letti dal file di configurazione \texttt{db.properties}, che permette di gestire profili multipli e di cambiare facilmente ambiente senza modificare il codice sorgente.

\subsubsection{Gestione delle risorse}
La classe \texttt{SupportDb} fornisce metodi di utilità per la chiusura sicura di oggetti JDBC come \texttt{Connection}, \texttt{Statement} e \texttt{ResultSet}, riducendo il rischio di memory leak e semplificando la gestione delle risorse.

\subsubsection{Dipendenze e configurazione Maven}
Il file \texttt{pom.xml} include la dipendenza per il driver JDBC PostgreSQL (\texttt{org.postgresql:postgresql}), necessaria per la comunicazione tra l'applicazione Java e il database. Inoltre, sono configurate le versioni di Java e i plugin Maven per la compilazione e l'esecuzione del progetto.

\subsubsection{Descrizione dei file principali}
\begin{itemize}
    \item \texttt{ConnectionJavaDb.java}: gestisce la lettura dei parametri di connessione dal file \texttt{db.properties}, carica il driver PostgreSQL e fornisce il metodo statico \texttt{getConnection()} per ottenere una connessione attiva al database. Supporta la selezione di profili multipli tramite la proprietà \texttt{db.profile}.
    \item \texttt{SupportDb.java}: fornisce metodi per la chiusura sicura di \texttt{Connection}, \texttt{Statement} e \texttt{ResultSet}, semplificando la gestione delle risorse JDBC e riducendo il rischio di errori.
    \item \texttt{pom.xml}: include la dipendenza per il driver JDBC PostgreSQL e configura la compilazione del progetto. La presenza della dipendenza \texttt{org.postgresql:postgresql} è fondamentale per la connessione al database.
\end{itemize}

\paragraph{Esempio di utilizzo}
\begin{verbatim}
// Ottenere una connessione al database
Connection conn = ConnectionJavaDb.getConnection();
// Eseguire una query
PreparedStatement ps = conn.prepareStatement("SELECT * FROM tabella");
ResultSet rs = ps.executeQuery();
// ...
// Chiusura delle risorse
new SupportDb().closeAll(conn, ps, rs);
\end{verbatim}

Questa architettura garantisce una connessione sicura, configurabile e facilmente estendibile tra l'applicazione Java e il database PostgreSQL.
\subsubsection{Best practice e vantaggi}
Questa architettura garantisce una connessione sicura, configurabile e facilmente estendibile tra l'applicazione Java e il database PostgreSQL. I vantaggi principali sono:
\begin{itemize}
    \item Centralizzazione della logica di accesso ai dati e gestione delle risorse.
    \item Facilità di manutenzione e aggiornamento delle dipendenze.
    \item Robustezza contro errori di connessione e memory leak.
    \item Portabilità e flessibilità grazie all'uso di JDBC e Maven.
    \item Codice più pulito e conforme alle best practice Java.
\end{itemize}
