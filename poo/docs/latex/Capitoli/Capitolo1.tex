\section{Introduzione}

Il progetto UninaFoodLab nasce con l'obiettivo di offrire una soluzione software moderna e intuitiva per la gestione di corsi culinari. La piattaforma è stata sviluppata seguendo le best practice adottando tecnologie consolidate come JavaFX per la realizzazione dell'interfaccia grafica e Maven per la gestione delle dipendenze e dei processi di build.
Il progetto è stato concepito per essere facilmente estendibile e manutenibile, grazie a una chiara separazione dei livelli logici.
La documentazione che segue illustra le scelte progettuali, le tecnologie utilizzate e le principali funzionalità implementate, con l'intento di fornire una panoramica completa e professionale del sistema sviluppato.
\subsection{Obiettivi del Progetto}
Il progetto UninaFoodLab si propone di:
\begin{itemize}
    \item Fornire una piattaforma intuitiva per la gestione di corsi culinari
    \item Semplificare la registrazione e la gestione degli utenti e per gli chef
    \item Facilitare la creazione e la gestione dei corsi, inclusa la pianificazione delle lezioni e la gestione delle ricette
    \item Semplificare la gestione delle prenotazioni e dei pagamenti per i corsi
\end{itemize}