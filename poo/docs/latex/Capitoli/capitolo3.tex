\section{Architettura del Progetto}

L'architettura del progetto UninaFoodLab è stata progettata per garantire una chiara separazione dei compiti e una facile manutenibilità. La struttura delle directory riflette i principali livelli logici dell'applicazione, secondo il pattern MVC e le best practice di progettazione.

Le principali suddivisioni sono:
\begin{itemize}
    \item \textbf{Boundary}: contiene le classi responsabili dell'interfaccia grafica e dell'interazione con l'utente, realizzate tramite JavaFX e FXML.
    \item \textbf{Controller}: gestisce la logica applicativa e il flusso degli eventi tra la GUI e i dati.
    \item \textbf{Entity}: suddivisa ulteriormente in \textit{DAO} (Data Access Object) e \textit{DTO} (Data Transfer Object). I DAO si occupano della persistenza e dell'accesso ai dati, mentre i DTO rappresentano le strutture dati scambiate tra i vari livelli.
    \item \textbf{JDBC}: contiene le classi e le utility per la connessione e la gestione del database PostgreSQL.
    \item \textbf{Utils}: raccoglie le classi di supporto e gli strumenti riutilizzabili all'interno del progetto.
\end{itemize}
Questa organizzazione favorisce la modularità e la scalabilità del sistema, permettendo di isolare le responsabilità e facilitare l'estensione futura. Ogni componente interagisce con gli altri tramite interfacce ben definite, riducendo le dipendenze e migliorando la qualità del codice.

La scelta di suddividere le entity in DAO e DTO consente di gestire in modo efficiente sia la persistenza che il trasferimento dei dati, mentre la presenza di una directory dedicata alle utility semplifica la gestione delle funzionalità trasversali.

\subsection{Struttura delle Directory}
La struttura delle directory del progetto è organizzata come segue:
\begin{itemize}
    \item \textbf{src/main/java}: contiene il codice sorgente dell'applicazione.
    \item \textbf{src/main/resources}: contiene le risorse dell'applicazione, come file FXML e immagini.
    \item \textbf{src/test/java}: contiene i test automatizzati.
\end{itemize}

\subsection{Class Diagram}

Il class diagram completo del progetto UninaFoodLab è stato realizzato e si trova all'interno della cartella \texttt{class diagram} del repository. Tuttavia, per motivi di leggibilità e chiarezza, non è stato inserito direttamente nella documentazione principale. 
La scelta di non includere il class diagram nel corpo del documento è dovuta alla sua notevole dimensione e complessità: il diagramma rappresenta tutte le classi, le relazioni e le dipendenze dell'intero sistema, risultando quindi molto esteso e di difficile consultazione su carta o in formato PDF. 
Per garantire comunque la completa trasparenza e la possibilità di analisi dettagliata, il file del class diagram è stato reso disponibile separatamente nella cartella dedicata, così da poter essere consultato in formato digitale e, se necessario, ingrandito o esplorato nei dettagli tramite strumenti appositi.
