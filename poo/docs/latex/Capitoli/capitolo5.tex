\section{Utils}
Nel progetto UninaFoodLab, la componente \textbf{utils} raccoglie tutte le classi di utilità e supporto che facilitano la gestione di operazioni comuni, la validazione dei dati, la manipolazione delle immagini, la gestione delle scene e il feedback all'utente. Questi utility sono pensati per essere riutilizzabili, modulari e indipendenti dalla logica di business, contribuendo a mantenere il codice pulito e manutenibile.

\subsection{Principali utility utilizzati}
Di seguito vengono descritti i principali utility implementati e utilizzati nel progetto:
\begin{itemize}
    \item \texttt{CardValidator}: gestisce la validazione delle carte di pagamento, controllando formato, tipo, scadenza e coerenza dei dati inseriti.
    \item \texttt{ImageClipUtils}: fornisce metodi per la manipolazione delle immagini profilo, come il ritaglio circolare e la gestione delle preview.
    \item \texttt{SceneSwitcher}: semplifica la navigazione tra le diverse scene JavaFX, gestendo il cambio pagina, il passaggio di parametri e la visualizzazione di dialog.
    \item \texttt{SuccessDialogUtils}: utility per la visualizzazione di dialog di conferma, successo o errore, con feedback visivo all'utente.
    \item \texttt{FrequenzaSessioniProvider}: fornisce in modo centralizzato le opzioni di frequenza delle sessioni di corso, recuperandole dal database e gestendo la cache interna.
    \item \texttt{UnifiedRecipeIngredientUI}: genera dinamicamente la UI per la gestione di ricette e ingredienti, offrendo modularità e notifiche automatiche al controller.
    \item \texttt{ErrorCaricamentoPropic}: gestisce le eccezioni relative al caricamento delle immagini profilo, fornendo messaggi di errore specifici e facilitando il debug.
\end{itemize}
Ognuna di queste classi è stata progettata per essere facilmente estendibile e integrabile nelle varie componenti dell'applicazione, favorendo la separazione delle responsabilità e la riusabilità del codice.

\subsection{Vantaggi dell'utilizzo di utility}
L'uso di utility nel progetto UninaFoodLab offre diversi vantaggi:
\begin{itemize}
    \item \textbf{Modularità}: le utility sono indipendenti dalla logica di business, consentendo di concentrarsi su operazioni specifiche senza appesantire le classi principali.
    \item \textbf{Riutilizzabilità}: le stesse utility possono essere utilizzate in diverse parti dell'applicazione, riducendo la duplicazione del codice e migliorando la manutenibilità.
    \item \textbf{Chiarezza del codice}: separando le operazioni comuni in classi dedicate, il codice principale risulta più leggibile e comprensibile, facilitando la collaborazione tra sviluppatori.
    \item \textbf{Testabilità}: le utility possono essere testate in modo indipendente, garantendo che funzionino correttamente senza dipendere da altre parti del sistema.
\end{itemize}
Questa organizzazione contribuisce a creare un'applicazione robusta, scalabile e facile da mantenere, rispettando le best practice di programmazione e design patterns.

\subsection{SceneSwitcher}
La classe \texttt{SceneSwitcher} è una utility fondamentale per la gestione della navigazione tra le diverse scene dell'applicazione JavaFX. Il suo obiettivo è centralizzare e semplificare il cambio pagina, la gestione delle dimensioni delle finestre, il passaggio di parametri tra controller e la visualizzazione di dialog personalizzati.

\paragraph{Ruolo e vantaggi}
\begin{itemize}
    \item Permette di passare da una scena all'altra in modo sicuro e modulare, evitando duplicazione di codice e errori di gestione delle finestre.
    \item Gestisce automaticamente le dimensioni minime, massime e predefinite delle finestre in base al tipo di scena (login, registrazione, main app, transazioni, dialog).
    \item Supporta la visualizzazione di dialog modali (es. conferma logout, calendario lezioni) centrando la finestra rispetto al parent.
    \item Consente il passaggio di parametri e controller tra le scene, facilitando la comunicazione tra le diverse componenti.
    \item Gestisce le differenze tra sistemi operativi (Windows, Linux, Mac) per massimizzazione e stile delle finestre.
\end{itemize}

\paragraph{Funzionalità principali}
\begin{itemize}
    \item \texttt{switchScene}: metodo generico per cambiare scena in base al tipo di pagina (login, register, main, payment, ecc.).
    \item \texttt{switchToLogin}, \texttt{switchToRegister}, \texttt{switchToMainApp}, \texttt{switchToTransaction}: metodi specifici per gestire le principali scene dell'applicazione.
    \item \texttt{showDialogCentered}: visualizza dialog modali centrati rispetto alla finestra principale.
    \item \texttt{showCalendarDialog}, \texttt{showLogoutDialog}: gestiscono la visualizzazione di dialog specializzati (calendario lezioni, conferma logout) con passaggio di parametri e controller.
    \item Gestione automatica delle dimensioni e massimizzazione delle finestre in base al contesto e al sistema operativo.
\end{itemize}

\paragraph{Esempio di utilizzo}
\begin{verbatim}
// Cambio scena verso la homepage chef
SceneSwitcher.switchScene(stage, "/fxml/homepagechef.fxml", "UninaFoodLab - Homepage Chef");

// Visualizzazione dialog di logout
LogoutDialogBoundary dialog = SceneSwitcher.showLogoutDialog(stage);
if (dialog.isConfirmed()) {
    SceneSwitcher.switchToLogin(stage, "/fxml/loginpage.fxml", "UninaFoodLab - Login");
}
\end{verbatim}

\paragraph{Integrazione e modularità}
SceneSwitcher è utilizzata in tutte le boundary e controller dell'applicazione per garantire una navigazione coerente, sicura e facilmente estendibile. La sua modularità consente di aggiungere nuove scene o dialog senza modificare la logica delle pagine principali, favorendo la manutenibilità e la scalabilità del progetto.

\subsection{SuccessDialogUtils}
La classe \texttt{SuccessDialogUtils} è una utility dedicata alla visualizzazione di dialog di conferma, successo o annullamento, offrendo feedback visivo e immediato all'utente in seguito a operazioni importanti (salvataggio dati, pagamento, annullamento modifiche, ecc.).

\paragraph{Ruolo e vantaggi}
\begin{itemize}
    \item Centralizza la gestione dei dialog di successo, evitando duplicazione di codice e garantendo coerenza grafica e funzionale.
    \item Permette di mostrare messaggi personalizzati in base al contesto (es. pagamento completato, dati salvati, modifiche annullate).
    \item Supporta la visualizzazione di informazioni aggiuntive (es. nome corso appena acquistato) e la personalizzazione dei titoli e dei messaggi.
    \item Gestisce la posizione e lo stile del dialog, centrando la finestra rispetto al parent e utilizzando uno stile grafico dedicato (\texttt{successdialog.fxml}).
    \item Facilita l'integrazione con boundary e controller, permettendo di mostrare feedback all'utente con una sola chiamata.
\end{itemize}

\paragraph{Funzionalità principali}
\begin{itemize}
    \item \texttt{showPaymentSuccessDialog}: mostra un dialog di conferma pagamento, con nome corso e messaggio personalizzato.
    \item \texttt{showSaveSuccessDialog}: mostra un dialog di conferma salvataggio dati.
    \item \texttt{showCancelSuccessDialog}: mostra un dialog di conferma annullamento modifiche.
    \item \texttt{showGenericSuccessDialog}: permette di mostrare dialog di successo generici, personalizzando titolo e messaggio.
    \item Gestione automatica della posizione, stile e chiusura del dialog.
\end{itemize}

\paragraph{Esempio di utilizzo}
\begin{verbatim}
// Dopo il salvataggio dei dati account
SuccessDialogUtils.showSaveSuccessDialog(stage);

// Dopo il pagamento di un corso
SuccessDialogUtils.showPaymentSuccessDialog(stage, "Corso di Sushi");

// Dopo l'annullamento delle modifiche
SuccessDialogUtils.showCancelSuccessDialog(stage);
\end{verbatim}

\paragraph{Integrazione e modularità}
SuccessDialogUtils è utilizzata in boundary e controller per offrire feedback immediato e coerente all'utente, migliorando l'esperienza d'uso e la chiarezza delle operazioni. La modularità della classe consente di estendere facilmente i dialog per nuovi contesti o messaggi, mantenendo la coerenza grafica e funzionale in tutta l'applicazione.

\subsection{CardValidator}
La classe \texttt{CardValidator} è una utility dedicata alla validazione delle carte di pagamento inserite dagli utenti. Il suo scopo è garantire che i dati relativi alle carte siano corretti, coerenti e conformi ai criteri di sicurezza richiesti per le transazioni online.

\paragraph{Ruolo e vantaggi}
\begin{itemize}
    \item Centralizza la logica di validazione delle carte, evitando duplicazione di codice e possibili errori nei controller.
    \item Permette di identificare il tipo di carta (Visa, Mastercard) in base al numero inserito, migliorando l'esperienza utente e la sicurezza.
    \item Verifica la validità del formato e della scadenza della carta, impedendo l'inserimento di dati errati o scaduti.
    \item Facilita l'integrazione con boundary e controller, permettendo di validare i dati con una sola chiamata.
\end{itemize}

\paragraph{Funzionalità principali}
\begin{itemize}
    \item \texttt{getCardType}: identifica il tipo di carta (Visa, Mastercard, Unknown) in base al prefisso del numero.
    \item \texttt{isValidCardType}: verifica se il numero inserito corrisponde a un tipo di carta supportato.
    \item \texttt{isValidExpiryDate}: controlla che la data di scadenza sia nel formato corretto (MM/YY o MM/YYYY) e che la carta non sia scaduta.
\end{itemize}

\paragraph{Criteri di validazione}
\begin{itemize}
    \item \textbf{Tipo carta}: accetta solo Visa (prefisso 4) e Mastercard (prefisso 5 o 2221-2720).
    \item \textbf{Formato scadenza}: accetta MM/YY o MM/YYYY, verifica che il mese sia tra 1 e 12 e che la data non sia passata.
    \item \textbf{Sicurezza}: impedisce l'inserimento di carte scadute o con formato non valido.
\end{itemize}

\paragraph{Esempio di utilizzo}
\begin{verbatim}
// Validazione tipo carta
if (!CardValidator.isValidCardType(cardNumber)) {
    showError("Tipo di carta non supportato.");
}

// Validazione scadenza
if (!CardValidator.isValidExpiryDate(expiry)) {
    showError("Data di scadenza non valida o carta scaduta.");
}
\end{verbatim}

\paragraph{Integrazione e modularità}
CardValidator è utilizzata in tutte le boundary e controller che gestiscono l'inserimento di carte di pagamento, garantendo coerenza, sicurezza e semplicità di utilizzo. La modularità della classe consente di estendere facilmente i criteri di validazione per supportare nuovi tipi di carte o regole di business.

\subsection{ImageClipUtils}
La classe \texttt{ImageClipUtils} è una utility dedicata alla manipolazione delle immagini profilo all'interno dell'applicazione. Il suo scopo principale è applicare un ritaglio circolare alle immagini, migliorando l'estetica della UI e garantendo uniformità nella visualizzazione delle foto utente e chef.

\paragraph{Ruolo e vantaggi}
\begin{itemize}
    \item Centralizza la logica di ritaglio delle immagini, evitando duplicazione di codice e garantendo coerenza grafica.
    \item Permette di applicare un clip circolare centrato a qualsiasi \texttt{ImageView}, con raggio personalizzabile o calcolato automaticamente.
    \item Migliora l'esperienza utente, offrendo una visualizzazione moderna e professionale delle immagini profilo.
    \item Facilita l'integrazione con boundary e controller, permettendo di applicare il ritaglio con una sola chiamata.
\end{itemize}

\paragraph{Funzionalità principali}
\begin{itemize}
    \item \texttt{setCircularClip}: applica un clip circolare centrato all'\texttt{ImageView} passato, con raggio specificato o calcolato in base alle dimensioni dell'immagine.
\end{itemize}

\paragraph{Esempio di utilizzo}
\begin{verbatim}
// Applica ritaglio circolare all'immagine profilo utente
ImageClipUtils.setCircularClip(userProfileImage, 40);

// Applica ritaglio automatico (metà lato minore)
ImageClipUtils.setCircularClip(profileImageLarge, 0);
\end{verbatim}

\paragraph{Integrazione e modularità}
ImageClipUtils è utilizzata in tutte le boundary che gestiscono immagini profilo (utente, chef), garantendo coerenza grafica e semplicità di utilizzo. La modularità della classe consente di estendere facilmente le funzionalità per nuovi tipi di ritaglio o effetti grafici.

\subsection{FrequenzaSessioniProvider}
La classe \texttt{FrequenzaSessioniProvider} è una utility pensata per gestire in modo centralizzato la fornitura delle opzioni di frequenza delle sessioni di corso. Il suo scopo è rendere disponibili, in modo efficiente e riutilizzabile, tutte le possibili frequenze (es. settimanale, bisettimanale, mensile) utilizzate nella creazione e modifica dei corsi.

\paragraph{Ruolo e vantaggi}
\begin{itemize}
    \item Centralizza la logica di recupero delle frequenze, evitando duplicazione di codice nei controller e boundary.
    \item Recupera le opzioni dal database tramite la DAO dedicata (\texttt{BarraDiRicercaDao}), garantendo che siano sempre aggiornate e coerenti con i dati reali.
    \item Utilizza una cache interna per evitare chiamate ripetute al database e migliorare le performance.
    \item Facilita l'integrazione con le form di creazione e modifica corso, permettendo di popolare le \texttt{ComboBox} con una sola chiamata.
\end{itemize}

\paragraph{Funzionalità principali}
\begin{itemize}
    \item \texttt{getFrequenze}: metodo statico che restituisce la lista delle frequenze disponibili, recuperandole dal database solo al primo utilizzo.
\end{itemize}

\paragraph{Esempio di utilizzo}
\begin{verbatim}
// Popola la ComboBox delle frequenze nella form di creazione corso
frequenzaComboBox.getItems().addAll(FrequenzaSessioniProvider.getFrequenze());
\end{verbatim}

\paragraph{Integrazione e modularità}
FrequenzaSessioniProvider è utilizzata in tutte le boundary e controller che gestiscono la creazione e modifica dei corsi, garantendo coerenza, efficienza e semplicità di utilizzo. La modularità della classe consente di estendere facilmente la logica per supportare nuove tipologie di frequenza o fonti dati.

\subsection{UnifiedRecipeIngredientUI}
La classe \texttt{UnifiedRecipeIngredientUI} è una utility avanzata dedicata alla generazione dinamica della UI per la gestione di ricette e ingredienti all'interno delle form di creazione e modifica corso. Il suo scopo è offrire un'interfaccia modulare, riutilizzabile e facilmente estendibile per la visualizzazione, aggiunta, modifica e rimozione di ricette e ingredienti.

\paragraph{Ruolo e vantaggi}
\begin{itemize}
    \item Centralizza la logica di costruzione della UI per ricette e ingredienti, garantendo coerenza grafica e funzionale tra le diverse scene.
    \item Permette di gestire ricette e ingredienti in modo dinamico, con supporto a modalità ibride e personalizzazione delle unità di misura.
    \item Facilita l'integrazione con boundary e controller, notificando automaticamente le modifiche tramite callback dedicate.
    \item Migliora l'esperienza utente, offrendo una UI intuitiva, ordinata e facilmente estendibile.
\end{itemize}

\paragraph{Funzionalità principali}
\begin{itemize}
    \item \texttt{createUnifiedRecipeBox}: genera un container grafico (VBox) per una ricetta, con campi per nome, lista ingredienti, pulsanti di aggiunta/rimozione e gestione callback.
    \item \texttt{createUnifiedIngredientBox}: genera una riga grafica (HBox) per un ingrediente, con campi per nome, quantità, unità di misura e pulsante di rimozione.
    \item Supporto a unità di misura personalizzabili e modalità ibride (es. corsi con sessioni di tipo diverso).
    \item Notifica automatica delle modifiche al controller tramite callback (Runnable).
\end{itemize}

\paragraph{Esempio di utilizzo}
\begin{verbatim}
// Genera la UI per una ricetta e la aggiunge al container principale
VBox recipeBox = UnifiedRecipeIngredientUI.createUnifiedRecipeBox(ricetta, recipesList, container, isHybrid, notifyControllerOfChange, unitaDiMisuraList);
container.getChildren().add(recipeBox);
\end{verbatim}

\paragraph{Integrazione e modularità}
UnifiedRecipeIngredientUI è utilizzata in tutte le boundary che gestiscono la creazione e modifica di corsi, garantendo coerenza, efficienza e semplicità di utilizzo. La modularità della classe consente di estendere facilmente la UI per nuove funzionalità (es. gestione allergeni, filtri avanzati, validazioni aggiuntive) senza modificare la logica principale delle form.

\subsection{ErrorCarimentoPropic}
La classe \texttt{ErrorCaricamentoPropic} è una utility dedicata alla gestione delle eccezioni durante il caricamento delle immagini profilo (propic) degli utenti e chef. Estende \texttt{RuntimeException} e fornisce un messaggio di errore specifico quando la propic non viene trovata o caricata correttamente.

\paragraph{Ruolo e vantaggi}
\begin{itemize}
    \item Centralizza la gestione degli errori relativi al caricamento delle immagini profilo, rendendo il codice più chiaro e facilmente manutenibile.
    \item Permette di distinguere rapidamente i problemi di caricamento propic da altre eccezioni, facilitando il debug e la gestione dei casi limite.
    \item Migliora la robustezza dell'applicazione, consentendo di gestire in modo controllato la mancanza di immagini profilo e di fornire feedback appropriato all'utente.
\end{itemize}

\paragraph{Funzionalità principali}
\begin{itemize}
    \item Costruttore senza parametri che imposta il messaggio di errore a \texttt{"Propic non trovata"}.
    \item Può essere lanciata da boundary, controller o utility che gestiscono il caricamento delle immagini profilo.
\end{itemize}

\paragraph{Esempio di utilizzo}
\begin{verbatim}
// Nel caricamento dell'immagine profilo utente
if (immaginePropic == null) {
    throw new ErrorCaricamentoPropic();
}
\end{verbatim}

\paragraph{Integrazione e modularità}
ErrorCaricamentoPropic è utilizzata in tutte le componenti che gestiscono il caricamento delle immagini profilo, in particolare nelle boundary e nelle utility grafiche. La sua modularità consente di estendere facilmente la gestione degli errori per altri tipi di immagini o risorse, mantenendo la coerenza e la chiarezza nella gestione delle eccezioni.
