\section{Introduzione}
\subsection{Descrizione del progetto}
Il seguente progetto, denominato UninaFoodLab, nasce con l'obiettivo di progettare e implementare un sistema informativo per la gestione di corsi di cucina tematici. Il sistema si propone di supportare gli chef nella creazione e 
gestione di corsi articolati in sessioni teoriche e pratiche, facilitando al contempo l’iscrizione e la partecipazione degli utenti. \\
La documentazione che segue illustra in dettaglio tutte le fasi della progettazione, analizzando le scelte architetturali, i modelli concettuali e logici, e le soluzioni tecniche adottate per garantire la correttezza, la coerenza e l'efficienza del sistema.

\subsection{Analisi dei Requisiti rilenvanti per il Database}

I seguenti requisiti funzionali sono stati analizzati ai fini della progettazione della base di dati. Essi definiscono le informazioni da memorizzare e le relazioni tra le entità del sistema.

\begin{enumerate}
    \item Registrazione e autenticazione degli utenti.
    \begin{itemize}
        \item Gli utenti devono poter creare un account con email e password.
        \item Gli utenti devono poter effettuare il login.
        \item Gli utenti devono poter recuperare la password dimenticata.
    \end{itemize}

    \item Gestione dei profili utente.
    \begin{itemize}
        \item Gli utenti devono poter visualizzare e modificare le proprie informazioni personali.
        \item Gli utenti devono poter visualizzare i propri corsi.
        \item Gli utenti devono poter visualizzare i propri dati di pagamento.
        \item Gli utenti devono poter visualizzare le proprie sessioni pratiche.
    \end{itemize}

    \item Creazione e gestione dei corsi da parte degli chef.
    \begin{itemize}
        \item Gli chef devono poter creare nuovi corsi.
        \item Gli chef devono poter definire le sessioni teoriche e pratiche.
        \item Gli chef devono poter modificare o cancellare i corsi.
    \end{itemize}

    \item Iscrizione ai corsi da parte degli utenti.
    \begin{itemize}
        \item Gli utenti devono poter consultare l’elenco dei corsi disponibili.
        \item Gli utenti devono poter iscriversi a un corso.
        \item Gli utenti devono ricevere conferma dell’iscrizione.
    \end{itemize}

    \item Gestione delle presenze e dei pagamenti.
    \begin{itemize}
        \item Gli chef devono poter registrare la presenza alle sessioni.
        \item Il sistema deve gestire i pagamenti degli utenti.
        \item Gli utenti devono poter visualizzare lo stato dei loro pagamenti.
    \end{itemize}
\end{enumerate}

