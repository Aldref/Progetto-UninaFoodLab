\section{Introduzione}
\subsection{Descrizione del progetto}
Il seguente progetto, denominato UninaFoodLab, nasce con l'obiettivo di progettare e implementare un sistema informativo per la gestione di corsi di cucina tematici. Il sistema si propone di supportare gli chef nella creazione e 
gestione di corsi articolati in sessioni teoriche e pratiche, facilitando al contempo l’iscrizione e la partecipazione degli utenti. \\
La documentazione che segue illustra in dettaglio tutte le fasi della progettazione, analizzando le scelte architetturali, i modelli concettuali e logici, e le soluzioni tecniche adottate per garantire la correttezza, la coerenza e l'efficienza del sistema.

\subsection{Analisi dei Requisiti rilenvanti per il Database}

I seguenti requisiti funzionali sono stati analizzati ai fini della progettazione della base di dati. Essi definiscono le informazioni da memorizzare e le relazioni tra le entità del sistema.

\begin{enumerate}
    \item Registrazione e autenticazione degli utenti.
    \begin{itemize}
        \item Gli utenti devono poter creare un account con email e password.
        \item Gli utenti devono poter effettuare il login.
    \end{itemize}

    \item Gestione dei profili dei partecipanti.
    \begin{itemize}
        \item I partecipanti devono poter visualizzare e modificare le proprie informazioni personali.
        \item I partecipanti devono poter visualizzare i propri corsi.
        \item I partecipanti devono poter visualizzare i propri dati di pagamento.
        \item I partecipanti devono poter visualizzare le proprie sessioni pratiche.
    \end{itemize}

    \item Creazione e gestione dei corsi da parte degli chef.
    \begin{itemize}
        \item Gli chef devono poter creare nuovi corsi.
        \item Gli chef devono poter definire le sessioni teoriche e pratiche.
        \item Gli chef devono poter modificare.
        \item Gli chef possono aggiungere delle ricette alle sessioni pratiche.
        \item Gli chef devono poter visualizzare il numero di ingredienti necessari per la sessione pratica
        \item Gli chef devono poter visualizzare il loro report di guadagno e attività.
    \end{itemize}

    \item Iscrizione ai corsi da parte dei partecipanti.
    \begin{itemize}
        \item I partecipanti devono poter consultare l’elenco dei corsi disponibili.
        \item I partecipanti devono poter iscriversi a un corso.
        \item I partecipanti devono dare conferma per la partecipazione alle sessioni pratiche.
    \end{itemize}

    \item Gestione delle presenze e dei pagamenti.
    \begin{itemize}
        \item Gli chef devono poter registrare la presenza alle sessioni.
        \item Il sistema deve gestire i pagamenti dei partecipanti.
        \item I partecipanti devono poter vedere le proprie carte di credito.
    \end{itemize}
\end{enumerate}

