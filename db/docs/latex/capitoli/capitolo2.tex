\section{progettazione concettuale}
\subsection{Introduzione}
Il modello concettuale rappresenta la struttura logica del database, definendo le entità, gli attributi e le relazioni tra di esse. In questa fase, si è proceduto a identificare le principali entità del sistema e a stabilire le relazioni che le collegano, garantendo così una visione chiara e coerente delle informazioni da gestire.
\subsection{UML non ristrutturato}
\begin{figure}[H]
    \noindent\makebox[\linewidth]{%
        \includegraphics[height=0.9\textheight,width=2\textwidth]{latex/immagini/uml_non_ristrutturato.pdf}
    }
    \caption{Diagramma UML del sistema}
\end{figure}
\subsubsection{Entità principali}
Le entità principali identificate nel sistema sono:
\begin{itemize}
    \item \textbf{Utente}: L’utente rappresenta il soggetto fruitore del sistema, che può iscriversi ai corsi e partecipare alle sessioni. I principali attributi includono nome, cognome, email, password, telefono, data di nascita e una foto di profilo. Ogni utente può essere associato a una o più carte di pagamento e può diventare partecipante a diversi corsi.
    \item \textbf{Chef}: Lo chef è un utente con il ruolo specifico di organizzare corsi. Ogni chef dispone di una descrizione e di un numero di anni di esperienza. Un chef può gestire più corsi, ma ogni corso è gestito da un solo chef.
    \item \textbf{Corso}: Il corso è l'entità centrale del sistema e rappresenta una proposta didattica su un tema gastronomico specifico. Contiene attributi quali nome, descrizione, identificativo, data di inizio/fine, frequenza delle sessioni, prezzo, immagine di copertina e tipo di cucina (modellato come enumerazione). Ogni corso è composto da più sessioni e prevede una relazione molti-a-molti con i partecipanti.
    \item \textbf{Sessione}: Ogni corso è articolato in una o più sessioni, ciascuna delle quali ha una data, un orario, un insieme di giorni della settimana in cui si svolge, e una durata. Le sessioni sono specializzate in due sottotipi mutuamente esclusivi:
    \begin{itemize}
        \item \textbf{Presenza}: Con attributi come luogo e attrezzature richieste.
        \item \textbf{Telematica}: Con attributi relativi all'app utilizzata e al codice di accesso.
    \end{itemize}
    \item \textbf{Partecipante e Adesione}: La partecipazione ai corsi è modellata tramite l’entità Partecipante, che collega utenti e corsi. La partecipazione a sessioni pratiche richiede un’adesione esplicita, rappresentata dall'entità Adesione, che contiene un attributo booleano di conferma.
    \item \textbf{Ricetta e Ingredientemento}: Ogni sessione pratica può includere la preparazione di una o più ricette. Ogni ricetta è composta da uno o più ingredienti, ciascuno dei quali ha un nome, una quantità e un'unità di misura (enumerata). La relazione tra Ricetta e Ingrediente è associativa e include l'attributo QuantitàTotale, utile per calcolare la quantità necessaria in base alle adesioni.
    \item \textbf{Carta e RichiestaPagamento}: Gli utenti possono associare al proprio profilo una o più carte di pagamento, appartenenti a un circuito specificato tramite enumerazione (Visa, Mastercard). Le richieste di pagamento sono entità separate, con data, stato (in attesa, pagato, fallito) e importo.
\end{itemize}
\subsubsection{Gerarchie e generalizzazioni}
Nel modello concettuale, sono state identificate le seguenti gerarchie e generalizzazioni:
\begin{itemize}
    \item \textbf{Sessione}: Le sessioni sono suddivise in due sottotipi: \textit{Presenza} e \textit{Telematica}. Questa specializzazione consente di gestire le specificità di ciascun tipo di sessione, come il luogo e le attrezzature per le sessioni in presenza, e l'app utilizzata e il codice di accesso per quelle telematiche.
    \item \textbf{Utente}: L'entità Utente può essere specializzata in due sottotipi: \textit{Partecipante} e \textit{Chef}. Questa distinzione permette di gestire le diverse funzionalità e attributi associati a ciascun ruolo nel sistema.
\end{itemize}
Entrambe le specializzazioni sono totali e disgiunte, di conseguenza ogni istanza di Sessione sia esclusivamente di uno dei due tipi e che ogni Utente sia o un Partecipante o uno Chef, ma non entrambi contemporaneamente.
\subsubsection{Relazioni tra le entità}
Le relazioni tra le entità sono state definite come segue:
\begin{itemize}
    \item \textbf{Utente - Partecipante}: Un utente può essere un partecipante a più corsi, e ogni corso può avere più partecipanti. Questa relazione è molti-a-molti.
    \item \textbf{Chef - Corso}: Ogni chef può gestire più corsi, ma ogni corso è associato a un solo chef. Questa relazione è uno-a-molti.
    \item \textbf{Corso - Sessione}: Un corso può avere più sessioni, ma ogni sessione appartiene a un solo corso. Questa relazione è uno-a-molti.
    \item \textbf{Sessione - Partecipante}: Ogni partecipante può aderire a più sessioni pratiche, e ogni sessione può avere più partecipanti. Questa relazione è molti-a-molti, mediata dall'entità Adesione.
    \item \textbf{Corso - Ricetta}: Ogni corso può includere più ricette, e ogni ricetta può essere associata a più corsi. Questa relazione è molti-a-molti.
    \item \textbf{Ricetta - Ingrediente}: Ogni ricetta può includere più ingredienti, e ogni ingrediente può essere utilizzato in più ricette. Questa relazione è molti-a-molti, mediata dall'attributo QuantitàTotale.
    \item \textbf{Utente - Carta}: Un utente può avere più carte di pagamento associate al proprio profilo. Questa relazione è uno-a-molti.
    \item \textbf{Ricetta - Ingrediente}: Ogni ricetta può essere associata a più ingredienti, e ogni ingrediente può essere utilizzato in più ricette. Questa relazione è una composizione, mediata dall'attributo QuantitàTotale.
\end{itemize}
\subsubsection{Motivazione delle scelte progettuali}
Le scelte progettuali sono state guidate dalla necessità di garantire una rappresentazione chiara e coerente delle informazioni, facilitando la gestione dei corsi, delle sessioni e delle partecipazioni. La specializzazione delle sessioni in Presenza e Telematica consente di gestire le specificità di ciascun tipo di sessione, mentre la distinzione tra Partecipante e Chef permette di differenziare i ruoli degli utenti nel sistema. Inoltre, l'uso di relazioni molti-a-molti per gestire le adesioni alle sessioni pratiche e le associazioni tra ricette e ingredienti garantisce flessibilità e scalabilità nel modello.

