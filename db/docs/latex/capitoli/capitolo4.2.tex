\subsection{Definizione dei Trigger}

\subsubsection{Trigger: Unicità Email tra Chef e Partecipante}

Questo trigger garantisce che lo stesso indirizzo email non possa essere usato contemporaneamente da un Chef e da un Partecipante. La funzione viene richiamata sia sulle tabelle Chef che Partecipante, in inserimento e aggiornamento.

\noindent\rule{\textwidth}{0.4pt}
\begin{lstlisting}[language=SQL, style=sqlstyle, literate={à}{{\`a}}1 {è}{{\`e}}1 {é}{{\'e}}1 {ì}{{\`i}}1 {ò}{{\`o}}1 {ù}{{\`u}}1]
CREATE OR REPLACE FUNCTION verifica_unicita_email()
RETURNS TRIGGER AS $$
BEGIN
    IF TG_TABLE_NAME = 'chef' THEN
        IF EXISTS (SELECT 1 FROM partecipante WHERE email = NEW.email) THEN
            RAISE EXCEPTION 'Errore: L''email "%" è già utilizzata da un Partecipante.', NEW.email;
        END IF;
    ELSIF TG_TABLE_NAME = 'partecipante' THEN
        IF EXISTS (SELECT 1 FROM chef WHERE email = NEW.email) THEN
            RAISE EXCEPTION 'Errore: L''email "%" è già utilizzata da uno Chef.', NEW.email;
        END IF;
    END IF;
    RETURN NEW;
END;
$$ LANGUAGE plpgsql;

CREATE TRIGGER trg_verifica_email_partecipante
BEFORE INSERT OR UPDATE ON partecipante
FOR EACH ROW
EXECUTE FUNCTION verifica_unicita_email();

CREATE TRIGGER trg_verifica_email_chef
BEFORE INSERT OR UPDATE ON chef
FOR EACH ROW
EXECUTE FUNCTION verifica_unicita_email();
\end{lstlisting}
\noindent\rule{\textwidth}{0.4pt}

\textbf{Spiegazione:}
\begin{itemize}
    \item \texttt{IF TG\_TABLE\_NAME = 'chef'}: Se il trigger è attivato dalla tabella Chef, controlla che l'email inserita o aggiornata non sia già presente nella tabella \texttt{partecipante} (campo \texttt{email}).
    \item \texttt{IF TG\_TABLE\_NAME = 'partecipante'}: Se il trigger è attivato dalla tabella Partecipante, controlla che l'email inserita o aggiornata non sia già presente nella tabella \texttt{chef} (campo \texttt{email}).
    \item \texttt{IF EXISTS (SELECT 1 ...)}: In entrambi i casi, se trova una corrispondenza, solleva un'eccezione e blocca l'inserimento/aggiornamento.
    \item \texttt{RAISE EXCEPTION}: Genera un errore se l'email è già in uso nell'altra tabella.
    \item Il trigger si attiva prima dell'inserimento o aggiornamento di un record (\texttt{BEFORE INSERT OR UPDATE}) sia su Chef che su Partecipante, garantendo l'unicità trasversale dell'email tra le due tabelle.
\end{itemize}

\subsubsection{Trigger: Controllo Formato Email}

Questo trigger verifica che l'indirizzo email inserito per Chef e Partecipante sia formalmente valido, ovvero contenga una chiocciola (@) e almeno un punto dopo la chiocciola, garantendo la correttezza sintattica degli indirizzi email memorizzati nel sistema.

\noindent\rule{\textwidth}{0.4pt}
\begin{lstlisting}[language=SQL, style=sqlstyle, literate={à}{{\`a}}1 {è}{{\`e}}1 {é}{{\'e}}1 {ì}{{\`i}}1 {ò}{{\`o}}1 {ù}{{\`u}}1]
CREATE OR REPLACE FUNCTION validate_email_full()
RETURNS TRIGGER AS $$
DECLARE
    at_pos INT;
    domain_part TEXT;
BEGIN
    at_pos := POSITION('@' IN NEW.Email);

    IF at_pos = 0 THEN
        RAISE EXCEPTION 'Email non valida: manca la chiocciola (@).';
    END IF;

    domain_part := SUBSTRING(NEW.Email FROM at_pos + 1);

    IF POSITION('.' IN domain_part) = 0 THEN
        RAISE EXCEPTION 'Email non valida: manca il punto dopo la chiocciola.';
    END IF;

    RETURN NEW;
END;
$$ LANGUAGE plpgsql;

CREATE TRIGGER trg_validate_email_partecipante
BEFORE INSERT OR UPDATE ON PARTECIPANTE
FOR EACH ROW
EXECUTE FUNCTION validate_email_full();

CREATE TRIGGER trg_validate_email_chef
BEFORE INSERT OR UPDATE ON CHEF
FOR EACH ROW
EXECUTE FUNCTION validate_email_full();
\end{lstlisting}
\noindent\rule{\textwidth}{0.4pt}

\textbf{Spiegazione:}
\begin{itemize}
    \item Il trigger si attiva prima di ogni inserimento o aggiornamento sulle tabelle \texttt{Chef} e \texttt{Partecipante}.
    \item La funzione controlla che l'email inserita contenga una chiocciola (@) e almeno un punto dopo la chiocciola.
    \item Se il formato non è valido, viene sollevata un'eccezione e l'operazione viene bloccata.
    \item In questo modo si garantisce che vengano accettati solo indirizzi email formalmente corretti.
\end{itemize}

\subsubsection{Trigger: Eliminazione Ingredienti Associati a Ricetta}

Quando una ricetta viene eliminata, questo trigger elimina automaticamente tutti gli ingredienti associati tramite la tabella PreparazioneIngrediente.

\noindent\rule{\textwidth}{0.4pt}
\begin{lstlisting}[language=SQL, style=sqlstyle, literate={à}{{\`a}}1 {è}{{\`e}}1 {é}{{\'e}}1 {ì}{{\`i}}1 {ò}{{\`o}}1 {ù}{{\`u}}1]
CREATE OR REPLACE FUNCTION elimina_ingredienti_della_ricetta()
RETURNS TRIGGER AS $$
BEGIN
    DELETE FROM PREPARAZIONEINGREDIENTE
    WHERE IdRicetta = OLD.IdRicetta;
    RETURN OLD;
END;
$$ LANGUAGE plpgsql;

CREATE TRIGGER trigger_elimina_ingredienti_associati
BEFORE DELETE ON RICETTA
FOR EACH ROW
EXECUTE FUNCTION elimina_ingredienti_della_ricetta();
\end{lstlisting}
\noindent\rule{\textwidth}{0.4pt}

\textbf{Spiegazione:}
\begin{itemize}
    \item \texttt{DELETE FROM PREPARAZIONEINGREDIENTE WHERE IdRicetta = OLD.IdRicetta}: Elimina tutti gli ingredienti collegati alla ricetta che sta per essere cancellata.
    \item Il trigger si attiva prima della cancellazione di una ricetta (\texttt{BEFORE DELETE ON RICETTA}).
    \item \texttt{RETURN OLD}: Permette la cancellazione della ricetta dopo aver eliminato i riferimenti.
\end{itemize}

\subsubsection{Trigger: Impedisci Inserimento di Carta Duplicata}

Questo trigger impedisce l'inserimento o l'aggiornamento di una carta di pagamento se esiste già una carta con le stesse ultime quattro cifre, data di scadenza, circuito e intestatario, evitando duplicati nel sistema.

\noindent\rule{\textwidth}{0.4pt}
\begin{lstlisting}[language=SQL, style=sqlstyle, literate={à}{{\`a}}1 {è}{{\`e}}1 {é}{{\'e}}1 {ì}{{\`i}}1 {ò}{{\`o}}1 {ù}{{\`u}}1]
CREATE OR REPLACE FUNCTION impedisci_carta_duplicata_per_partecipante()
RETURNS TRIGGER AS $$
DECLARE
    existing_card_id INT;
BEGIN
    IF TG_TABLE_NAME = 'carta' THEN
        IF EXISTS (
            SELECT 1
            FROM Carta
            WHERE UltimeQuattroCifre = NEW.UltimeQuattroCifre
                AND DataScadenza = NEW.DataScadenza
                AND Circuito = NEW.Circuito::VARCHAR(50)
                AND Intestatario = NEW.Intestatario
                AND IdCarta != NEW.IdCarta
        ) THEN
            RAISE EXCEPTION 'Errore: Una carta con queste ultime 4 cifre, data di scadenza, circuito e intestatario esiste già nel sistema.';
        END IF;
    END IF;
    RETURN NEW;
END;
$$ LANGUAGE plpgsql;

CREATE TRIGGER trg_impedisci_dettagli_carta_duplicati
BEFORE INSERT OR UPDATE ON Carta
FOR EACH ROW
EXECUTE FUNCTION impedisci_carta_duplicata_per_partecipante();
\end{lstlisting}
\noindent\rule{\textwidth}{0.4pt}

\textbf{Spiegazione:}
\begin{itemize}
    \item Il trigger si attiva prima di ogni inserimento o aggiornamento sulla tabella \texttt{Carta}.
    \item La funzione controlla se esiste già una carta con le stesse ultime quattro cifre, data di scadenza, circuito e intestatario, ma con un identificativo diverso.
    \item Se trova una corrispondenza, viene sollevata un'eccezione e l'operazione viene bloccata.
    \item Questo garantisce che non possano essere inserite carte duplicate nel sistema.
\end{itemize}

\subsubsection{Trigger: Massimo 2 Tipi di Cucina per Corso}

Questo trigger impedisce di associare più di due tipi di cucina a uno stesso corso, garantendo il rispetto del vincolo applicativo.

\noindent\rule{\textwidth}{0.4pt}
\begin{lstlisting}[language=SQL, style=sqlstyle, literate={à}{{\`a}}1 {è}{{\`e}}1 {é}{{\'e}}1 {ì}{{\`i}}1 {ò}{{\`o}}1 {ù}{{\`u}}1]
CREATE OR REPLACE FUNCTION verifica_max_tipi_cucina_per_corso()
RETURNS TRIGGER AS $$
DECLARE
    current_tipi_count INTEGER;
BEGIN
    SELECT COUNT(*)
    INTO current_tipi_count
    FROM TipoDiCucina_Corso
    WHERE IDCorso = NEW.IDCorso;

    IF current_tipi_count >= 2 THEN
        RAISE EXCEPTION 'Errore: Il corso con ID % ha già raggiunto il limite massimo di 2 tipi di cucina associati.', NEW.IDCorso;
    END IF;

    RETURN NEW;
END;
$$ LANGUAGE plpgsql;

CREATE TRIGGER trg_max_tipi_cucina_corso
BEFORE INSERT ON TipoDiCucina_Corso
FOR EACH ROW
EXECUTE FUNCTION verifica_max_tipi_cucina_per_corso();
\end{lstlisting}
\noindent\rule{\textwidth}{0.4pt}

\textbf{Spiegazione:}
\begin{itemize}
    \item Il trigger si attiva prima di ogni inserimento sulla tabella \texttt{TipoDiCucina\_Corso}.
    \item La funzione conta quanti tipi di cucina sono già associati al corso specificato da \texttt{NEW.IDCorso}.
    \item Se il numero è già 2 o più, viene sollevata un'eccezione e l'inserimento viene bloccato.
    \item In questo modo si garantisce che ogni corso possa avere al massimo due tipi di cucina associati, come richiesto dal vincolo applicativo.
\end{itemize}

\subsubsection{Trigger: Validazione Intervallo Date Corso}

Questo trigger garantisce che la data di inizio di un corso non sia successiva alla data di fine e che la data di inizio non sia antecedente alla data corrente, assicurando la coerenza temporale dei corsi inseriti o aggiornati.

\noindent\rule{\textwidth}{0.4pt}
\begin{lstlisting}[language=SQL, style=sqlstyle, literate={à}{{\`a}}1 {è}{{\`e}}1 {é}{{\'e}}1 {ì}{{\`i}}1 {ò}{{\`o}}1 {ù}{{\`u}}1]
CREATE OR REPLACE FUNCTION verifica_intervallo_date_corso()
RETURNS TRIGGER AS $$
BEGIN
    IF NEW.DataInizio > NEW.DataFine THEN
        RAISE EXCEPTION 'Errore: La DataInizio del corso (%) non può essere successiva alla DataFine (%).',
            NEW.DataInizio, NEW.DataFine;
    END IF;

    IF NEW.DataInizio < CURRENT_DATE THEN
        RAISE EXCEPTION 'Errore: La DataInizio del corso (%) non può essere antecedente alla data corrente (%).',
            NEW.DataInizio, CURRENT_DATE;
    END IF;

    RETURN NEW;
END;
$$ LANGUAGE plpgsql;

CREATE TRIGGER trg_valida_date_corso
BEFORE INSERT OR UPDATE ON Corso
FOR EACH ROW
EXECUTE FUNCTION verifica_intervallo_date_corso();
\end{lstlisting}
\noindent\rule{\textwidth}{0.4pt}

\textbf{Spiegazione:}
\begin{itemize}
    \item Il trigger si attiva prima di ogni inserimento o aggiornamento sulla tabella \texttt{Corso}.
    \item La funzione controlla che la data di inizio non sia successiva a quella di fine e che non sia antecedente alla data corrente.
    \item Se una delle due condizioni non è rispettata, viene sollevata un'eccezione e l'operazione viene bloccata.
    \item In questo modo si garantisce la coerenza temporale dei dati relativi ai corsi.
\end{itemize}

\subsubsection{Trigger: Importo Pagato Deve Corrispondere al Costo del Corso}

Questo trigger garantisce che l'importo pagato per una richiesta di pagamento corrisponda esattamente al prezzo del corso associato, evitando discrepanze tra quanto dovuto e quanto effettivamente pagato.

\noindent\rule{\textwidth}{0.4pt}
\begin{lstlisting}[language=SQL, style=sqlstyle, literate={à}{{\`a}}1 {è}{{\`e}}1 {é}{{\'e}}1 {ì}{{\`i}}1 {ò}{{\`o}}1 {ù}{{\`u}}1]
CREATE OR REPLACE FUNCTION verifica_importo_pagamento_corrisponde_prezzo_corso()
RETURNS TRIGGER AS $$
DECLARE
    course_price DECIMAL(10,2);
BEGIN
    SELECT Prezzo INTO course_price
    FROM Corso
    WHERE IdCorso = NEW.IdCorso;

    IF NEW.ImportoPagato != course_price THEN
        RAISE EXCEPTION 'Errore: L''ImportoPagato (%.2f) deve corrispondere esattamente al Prezzo del corso (%.2f).', NEW.ImportoPagato, course_price;
    END IF;

    RETURN NEW;
END;
$$ LANGUAGE plpgsql;

CREATE TRIGGER trg_valida_importo_pagamento
BEFORE INSERT OR UPDATE ON RichiestaPagamento
FOR EACH ROW
EXECUTE FUNCTION verifica_importo_pagamento_corrisponde_prezzo_corso();
\end{lstlisting}
\noindent\rule{\textwidth}{0.4pt}

\textbf{Spiegazione:}
\begin{itemize}
    \item Il trigger si attiva prima di ogni inserimento o aggiornamento sulla tabella \texttt{RichiestaPagamento}.
    \item La funzione recupera il prezzo del corso associato alla richiesta di pagamento tramite l'\texttt{IdCorso}.
    \item Se l'\texttt{ImportoPagato} non corrisponde esattamente al prezzo del corso, viene sollevata un'eccezione e l'operazione viene bloccata.
    \item In questo modo si garantisce che ogni pagamento sia coerente con il costo effettivo del corso.
\end{itemize}

\subsubsection{Trigger: Età Compresa tra 18 e 100 Anni}

Questo trigger garantisce che la data di nascita inserita per Chef e Partecipante produca un'età compresa tra 18 e 100 anni, assicurando che solo utenti con età valida possano essere registrati nel sistema.

\noindent\rule{\textwidth}{0.4pt}
\begin{lstlisting}[language=SQL, style=sqlstyle, literate={à}{{\`a}}1 {è}{{\`e}}1 {é}{{\'e}}1 {ì}{{\`i}}1 {ò}{{\`o}}1 {ù}{{\`u}}1]
CREATE OR REPLACE FUNCTION verifica_intervallo_eta()
RETURNS TRIGGER AS $$
DECLARE
    person_age_years INTEGER;
    min_age CONSTANT INTEGER := 18;
    max_age CONSTANT INTEGER := 100;
BEGIN
    IF NEW.DataDiNascita IS NULL THEN
        RAISE EXCEPTION 'Errore: La DataDiNascita non può essere NULL.';
    END IF;

    person_age_years := EXTRACT(YEAR FROM AGE(CURRENT_DATE, NEW.DataDiNascita));

    IF person_age_years < min_age THEN
        RAISE EXCEPTION 'Errore: L''età della persona (%, calcolata dalla DataDiNascita %) è inferiore all''età minima consentita (%).', person_age_years, NEW.DataDiNascita, min_age;
    END IF;

    IF person_age_years > max_age THEN
        RAISE EXCEPTION 'Errore: L''età della persona (%, calcolata dalla DataDiNascita %) è superiore all''età massima consentita (%).', person_age_years, NEW.DataDiNascita, max_age;
    END IF;

    RETURN NEW;
END;
$$ LANGUAGE plpgsql;

CREATE TRIGGER trg_valida_eta_chef
BEFORE INSERT OR UPDATE ON Chef
FOR EACH ROW
EXECUTE FUNCTION verifica_intervallo_eta();

CREATE TRIGGER trg_valida_eta_partecipante
BEFORE INSERT OR UPDATE ON Partecipante
FOR EACH ROW
EXECUTE FUNCTION verifica_intervallo_eta();
\end{lstlisting}
\noindent\rule{\textwidth}{0.4pt}

\textbf{Spiegazione:}
\begin{itemize}
    \item Il trigger si attiva prima di ogni inserimento o aggiornamento sulle tabelle \texttt{Chef} e \texttt{Partecipante}.
    \item La funzione calcola l'età a partire dalla data di nascita fornita.
    \item Se l'età è inferiore a 18 anni o superiore a 100 anni, viene sollevata un'eccezione e l'operazione viene bloccata.
    \item In questo modo si garantisce che solo utenti con età valida possano essere registrati nel sistema.
\end{itemize}

\subsubsection{Trigger: Controllo Superamento Numero Massimo di Partecipanti per Corso}

Questo trigger impedisce che il numero di partecipanti paganti a un corso superi il limite massimo definito dal campo \texttt{MaxPersone} della tabella \texttt{Corso}. In questo modo si garantisce che non vengano accettate richieste di pagamento che eccedono la capienza prevista per ciascun corso.

\noindent\rule{\textwidth}{0.4pt}
\begin{lstlisting}[language=SQL, style=sqlstyle, literate={à}{{\`a}}1 {è}{{\`e}}1 {é}{{\'e}}1 {ì}{{\`i}}1 {ò}{{\`o}}1 {ù}{{\`u}}1]
CREATE OR REPLACE FUNCTION verifica_superamento_max_partecipanti_corso()
RETURNS TRIGGER AS $$
DECLARE
    course_max_people INTEGER;
    current_paid_count INTEGER;
    potential_paid_count INTEGER;
    course_name VARCHAR(255);
BEGIN
    SELECT MaxPersone, Nome INTO course_max_people, course_name
    FROM Corso
    WHERE IdCorso = NEW.IdCorso;

    IF course_max_people <= 0 OR course_max_people IS NULL THEN
        RETURN NEW;
    END IF;

    SELECT COUNT(*)
    INTO current_paid_count
    FROM RichiestaPagamento
    WHERE IdCorso = NEW.IdCorso
        AND StatoPagamento = 'Pagato';

    potential_paid_count := current_paid_count;

    IF TG_OP = 'INSERT' THEN
        IF NEW.StatoPagamento = 'Pagato' THEN
            potential_paid_count := potential_paid_count + 1;
        END IF;
    ELSIF TG_OP = 'UPDATE' THEN
        IF OLD.StatoPagamento != 'Pagato' AND NEW.StatoPagamento = 'Pagato' THEN
            potential_paid_count := potential_paid_count + 1;
        ELSIF OLD.StatoPagamento = 'Pagato' AND NEW.StatoPagamento != 'Pagato' THEN
            potential_paid_count := potential_paid_count - 1;
        END IF;
    END IF;

    IF potential_paid_count > course_max_people THEN
        RAISE EXCEPTION 'Errore: Il corso "%" (ID %) ha raggiunto il limite massimo di % partecipanti paganti. Impossibile accettare questa richiesta di pagamento (attuali: %s, limite: %s).',
                            course_name, NEW.IdCorso, course_max_people, current_paid_count, course_max_people;
    END IF;

    RETURN NEW;
END;
$$ LANGUAGE plpgsql;

CREATE TRIGGER trg_controllo_max_partecipanti_corso
BEFORE INSERT OR UPDATE ON RichiestaPagamento
FOR EACH ROW
EXECUTE FUNCTION verifica_superamento_max_partecipanti_corso();
\end{lstlisting}
\noindent\rule{\textwidth}{0.4pt}

\textbf{Spiegazione:}
\begin{itemize}
    \item Il trigger si attiva prima di ogni inserimento o aggiornamento sulla tabella \texttt{RichiestaPagamento}.
    \item La funzione recupera il numero massimo di partecipanti paganti previsto per il corso e il numero attuale di pagamenti confermati.
    \item In caso di inserimento o aggiornamento che porterebbe il totale dei paganti oltre il limite, viene sollevata un'eccezione e l'operazione viene bloccata.
    \item In questo modo si garantisce il rispetto della capienza massima prevista per ciascun corso.
\end{itemize}

\subsubsection{Trigger: Impedire Eliminazione di Corsi con Iscritti Paganti}

Questo trigger impedisce l'eliminazione di un corso se sono ancora presenti partecipanti che hanno già effettuato il pagamento, garantendo l'integrità delle iscrizioni e la correttezza dei dati gestiti dal sistema.

\noindent\rule{\textwidth}{0.4pt}
\begin{lstlisting}[language=SQL, style=sqlstyle, literate={à}{{\`a}}1 {è}{{\`e}}1 {é}{{\'e}}1 {ì}{{\`i}}1 {ò}{{\`o}}1 {ù}{{\`u}}1]
CREATE OR REPLACE FUNCTION impedisci_eliminazione_corso_se_iscritto()
RETURNS TRIGGER AS $$
DECLARE
    enrolled_count INTEGER;
    course_name VARCHAR(255);
BEGIN
    SELECT Nome INTO course_name
    FROM Corso
    WHERE IdCorso = OLD.IdCorso;

    SELECT COUNT(*)
    INTO enrolled_count
    FROM RichiestaPagamento
    WHERE IdCorso = OLD.IdCorso
        AND StatoPagamento = 'Pagato';

    IF enrolled_count > 0 THEN
        RAISE EXCEPTION 'Errore: Impossibile eliminare il corso "%" (ID %). Ci sono ancora % partecipanti paganti iscritti.',
                            course_name, OLD.IdCorso, enrolled_count;
    END IF;

    RETURN OLD;
END;
$$ LANGUAGE plpgsql;

CREATE TRIGGER trg_impedisci_eliminazione_corso
BEFORE DELETE ON Corso
FOR EACH ROW
EXECUTE FUNCTION impedisci_eliminazione_corso_se_iscritto();
\end{lstlisting}
\noindent\rule{\textwidth}{0.4pt}

\textbf{Spiegazione:}
\begin{itemize}
    \item Il trigger si attiva prima della cancellazione di un corso dalla tabella \texttt{Corso}.
    \item La funzione verifica se esistono partecipanti paganti ancora iscritti al corso.
    \item Se il numero di iscritti paganti è maggiore di zero, viene sollevata un'eccezione e l'eliminazione viene bloccata.
    \item In questo modo si garantisce che non vengano eliminati corsi con iscritti attivi, preservando la coerenza dei dati.
\end{itemize}

\subsubsection{Trigger: Chef Non Può Essere Assegnato Contemporaneamente a Sessione Presenza e Telematica}

Questo trigger impedisce che uno stesso chef sia assegnato contemporaneamente, nello stesso giorno, sia a una sessione in presenza che a una sessione telematica, garantendo la coerenza degli orari.

\noindent\rule{\textwidth}{0.4pt}
\begin{lstlisting}[language=SQL, style=sqlstyle, literate={à}{{\`a}}1 {è}{{\`e}}1 {é}{{\'e}}1 {ì}{{\`i}}1 {ò}{{\`o}}1 {ù}{{\`u}}1]
CREATE OR REPLACE FUNCTION verifica_sessioni_contemporanee_chef()
RETURNS TRIGGER AS $$
DECLARE
    conflicting_session_count INTEGER;
    chef_name VARCHAR(255);
    session_type_being_inserted TEXT;
    other_session_type TEXT;
BEGIN
    SELECT Nome || ' ' || Cognome INTO chef_name
    FROM Chef
    WHERE IdChef = NEW.IDChef;

    IF TG_TABLE_NAME = 'sessione_presenza' THEN
        session_type_being_inserted := 'presenza';
        other_session_type := 'telematica';
    ELSIF TG_TABLE_NAME = 'sessione_telematica' THEN
        session_type_being_inserted := 'telematica';
        other_session_type := 'presenza';
    ELSE
        RAISE EXCEPTION 'Errore interno del trigger: tabella sconosciuta (%).', TG_TABLE_NAME;
    END IF;

    IF session_type_being_inserted = 'presenza' THEN
        SELECT COUNT(*)
        INTO conflicting_session_count
        FROM Sessione_Telematica
        WHERE IDChef = NEW.IDChef
            AND Data = NEW.Data;
    ELSIF session_type_being_inserted = 'telematica' THEN
        SELECT COUNT(*)
        INTO conflicting_session_count
        FROM Sessione_Presenza
        WHERE IDChef = NEW.IDChef
            AND Data = NEW.Data;
    END IF;

    IF conflicting_session_count > 0 THEN
        RAISE EXCEPTION 'Errore: Lo Chef "%" (ID %) è già assegnato a una sessione di tipo "%" in data %. Impossibile assegnarlo anche a una sessione di tipo "%" nello stesso giorno.',
                            chef_name, NEW.IDChef, other_session_type, NEW.Data, session_type_being_inserted;
    END IF;

    RETURN NEW;
END;
$$ LANGUAGE plpgsql;

CREATE TRIGGER trg_verifica_sessione_presenza_chef
BEFORE INSERT OR UPDATE ON Sessione_Presenza
FOR EACH ROW
EXECUTE FUNCTION verifica_sessioni_contemporanee_chef();

CREATE TRIGGER trg_verifica_sessione_telematica_chef
BEFORE INSERT OR UPDATE ON Sessione_Telematica
FOR EACH ROW
EXECUTE FUNCTION verifica_sessioni_contemporanee_chef();
\end{lstlisting}
\noindent\rule{\textwidth}{0.4pt}

\textbf{Spiegazione:}
\begin{itemize}
    \item Il trigger si attiva prima di ogni inserimento o aggiornamento sulle tabelle \texttt{Sessione\_Presenza} e \texttt{Sessione\_Telematica}.
    \item La funzione controlla che lo stesso chef non sia già assegnato a una sessione di tipo diverso nello stesso giorno.
    \item Se viene rilevata una sovrapposizione, viene sollevata un'eccezione e l'operazione viene bloccata.
    \item In questo modo si garantisce che uno chef non possa essere assegnato contemporaneamente a sessioni di tipo diverso nello stesso giorno.
\end{itemize}

\subsection{Definizione delle View}

\subsubsection{View: Statistiche Mensili Chef}

Questa view fornisce una panoramica mensile delle attività di ciascun chef, utile per la generazione del grafico del report mensile. Riassume il numero di ricette preparate, i valori massimo, minimo e medio di ricette per sessione, il numero di corsi e sessioni tenuti, e il guadagno totale del mese corrente.

\noindent\rule{\textwidth}{0.4pt}
\begin{lstlisting}[language=SQL, style=sqlstyle, literate={à}{{\`a}}1 {è}{{\`e}}1 {é}{{\'e}}1 {ì}{{\`i}}1 {ò}{{\`o}}1 {ù}{{\`u}}1]
CREATE OR REPLACE VIEW vista_statistiche_mensili_chef AS
  SELECT
    ch.IdChef,
    ch.Nome,
    ch.Cognome,
    COUNT(spr.IdRicetta) AS totale_ricette_mese,
    COALESCE(MAX(spr_count.ricette_per_sessione), 0) AS max_ricette_in_sessione,
    COALESCE(MIN(spr_count.ricette_per_sessione), 0) AS min_ricette_in_sessione,
    COALESCE(AVG(spr_count.ricette_per_sessione), 0) AS media_ricette_per_sessione,
    COUNT(DISTINCT c.IdCorso) AS numero_corsi,
    COUNT(DISTINCT sp.IdSessionePresenza) AS numero_sessioni_presenza,
    COUNT(DISTINCT st.IdSessioneTelematica) AS numero_sessioni_telematiche,
    COALESCE(SUM(rp.ImportoPagato), 0) AS guadagno_totale
    FROM CHEF ch
    LEFT JOIN CORSO c ON ch.IdChef = c.IdChef
    LEFT JOIN SESSIONE_PRESENZA sp ON c.IdCorso = sp.IDcorso AND sp.IDChef = ch.IdChef
    LEFT JOIN SESSIONE_PRESENZA_RICETTA spr ON spr.IdSessionePresenza = sp.IdSessionePresenza
    LEFT JOIN (
        SELECT spr2.IdSessionePresenza, COUNT(*) AS ricette_per_sessione
        FROM SESSIONE_PRESENZA_RICETTA spr2
        JOIN SESSIONE_PRESENZA sp2 ON spr2.IdSessionePresenza = sp2.IdSessionePresenza
        WHERE DATE_PART('month', sp2.Data) = DATE_PART('month', CURRENT_DATE)
        AND DATE_PART('year', sp2.Data) = DATE_PART('year', CURRENT_DATE)
        GROUP BY spr2.IdSessionePresenza
    ) spr_count ON spr_count.IdSessionePresenza = sp.IdSessionePresenza
    LEFT JOIN SESSIONE_TELEMATICA st ON c.IdCorso = st.IDcorso AND st.IDChef = ch.IdChef
        AND DATE_PART('month', st.Data) = DATE_PART('month', CURRENT_DATE)
        AND DATE_PART('year', st.Data) = DATE_PART('year', CURRENT_DATE)
    LEFT JOIN RICHIESTAPAGAMENTO rp ON rp.IdCorso = c.IdCorso
        AND DATE_PART('month', rp.DataRichiesta) = DATE_PART('month', CURRENT_DATE)
        AND DATE_PART('year', rp.DataRichiesta) = DATE_PART('year', CURRENT_DATE)
    WHERE
    (
    (sp.Data IS NULL OR (
        DATE_PART('month', sp.Data) = DATE_PART('month', CURRENT_DATE)
        AND DATE_PART('year', sp.Data) = DATE_PART('year', CURRENT_DATE)
    ))
    )
    GROUP BY ch.IdChef, ch.Nome, ch.Cognome;
\end{lstlisting}
\noindent\rule{\textwidth}{0.4pt}

\textbf{Spiegazione:}
\begin{itemize}
    \item La view aggrega i dati per ogni chef, limitando i risultati al mese e anno correnti tramite la funzione \texttt{DATE\_PART}.
    \item Calcola il totale delle ricette preparate nel mese, il massimo, minimo e media di ricette per sessione, il numero di corsi, sessioni in presenza e telematiche, e il guadagno totale.
    \item Utilizza le funzioni di aggregazione SQL come \texttt{COUNT}, \texttt{MAX}, \texttt{MIN}, \texttt{AVG}, \texttt{SUM} e \texttt{COALESCE} per gestire i casi in cui non ci siano dati (restituendo zero).
    \item Le join collegano chef, corsi, sessioni, ricette e pagamenti per ottenere una panoramica completa delle attività mensili di ciascun chef.
    \item Questa view è pensata per essere utilizzata come base dati per la generazione di report e grafici mensili sulle performance degli chef.
\end{itemize}

\subsubsection{View: Quantità Ingredienti per Sessione}

Questa view calcola dinamicamente la quantità totale necessaria di ciascun ingrediente per ogni sessione in presenza, in base al numero di partecipanti che hanno confermato la presenza. È utile per aggiornare automaticamente la quantità totale da preparare ogni volta che viene registrata una nuova adesione confermata.

\noindent\rule{\textwidth}{0.4pt}
\begin{lstlisting}[language=SQL, style=sqlstyle, literate={à}{{\`a}}1 {è}{{\`e}}1 {é}{{\'e}}1 {ì}{{\`i}}1 {ò}{{\`o}}1 {ù}{{\`u}}1]
CREATE OR REPLACE VIEW QuantitaPerSessione AS
  SELECT 
        pi.IdRicetta,
        pi.IdIngrediente,
        pi.QuanititaUnitaria,
        COUNT(asp.IDpartecipante) AS NumeroPartecipanti,
        pi.QuanititaUnitaria * COUNT(asp.IDpartecipante) AS QuantitaTotale,
        spr.Idsessionepresenza
    FROM PREPARAZIONEINGREDIENTE pi
    JOIN SESSIONE_PRESENZA_RICETTA spr ON pi.IdRicetta = spr.Idricetta
    JOIN ADESIONE_SESSIONEPRESENZA asp ON asp.Idsessionepresenza = spr.Idsessionepresenza
    WHERE asp.Conferma = true
    GROUP BY pi.IdRicetta, pi.IdIngrediente, pi.QuanititaUnitaria, spr.Idsessionepresenza;
\end{lstlisting}
\noindent\rule{\textwidth}{0.4pt}

\textbf{Spiegazione:}
\begin{itemize}
    \item La view collega le tabelle delle preparazioni ingredienti, delle sessioni in presenza e delle adesioni confermate.
    \item Per ogni ingrediente di ogni ricetta associata a una sessione, calcola il numero di partecipanti confermati e la quantità totale necessaria (quantità unitaria moltiplicata per il numero di partecipanti).
    \item Ogni volta che viene registrata una nuova adesione confermata, la view restituisce automaticamente il valore aggiornato della quantità totale da preparare.
    \item Utile per la gestione logistica e l'approvvigionamento degli ingredienti in base alle presenze effettive.
\end{itemize}